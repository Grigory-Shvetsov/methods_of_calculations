\documentclass[12pt, a4paper]{article}

\usepackage[utf8]{inputenc}
\usepackage[T1]{fontenc}
\usepackage[russian]{babel}
\usepackage[oglav,spisok,boldsect, figwhole]{./style/fn2kursstyle1}
\graphicspath{{./style/}{./figures/}}
%\usepackage{float}%"Плавающие" картинки
\usepackage{multirow}
\usepackage{subcaption}
%Римские цифры
\newcommand{\RomanNumeralCaps}[1]
{\MakeUppercase{\romannumeral #1}}

\usepackage{comment}

%Параметры титульника
\title{Итерационные методы решения СЛАУ}
\group{ФН2-52Б}
\author{Г.А.~Швецов}
\supervisor{А.О.~Гусев}
\date{2022}
\begin{document}
	\newcommand{\pl}{\partial}
	\maketitle
	
	\tableofcontents
	
	\newpage

	\section{Контрольные вопросы}
	% TODO %%%%%%%%%%%%%%%%%%%%%%%%%%%%%%%%%%%%%%%%%%%%%%%%%%%%
	\begin{enumerate}
		\item \textit{Почему условие $\|C\|<1$ гарантирует сходимость итерационных методов?}
		\smallskip
		\textbf{Доказательство:}
		
		Для итерационного метода
		\begin{equation}
			\label{sol}
			x = Cx + y,
		\end{equation}
		где $x$ --- решение соответствующей СЛАУ. На $k$-ом шаге:
		\begin{equation}
			\label{k_iter}
			x^{k+1} = C x^k + y
		\end{equation}
		
		Вычитая \eqref{k_iter} из \eqref{sol}, получаем:
		\begin{eqnarray*}
			& x - x^{k+1} = C(x - x^k); \\
			& \|x - x^{k+1}\| = \|C(x - x^k)\| \le \|C\| \, \|x - x^k\| \le \|C\|^2 \|x - x^{k-1}\| \le \dots \le \\
			& \le \|C\|^{k+1} \|x - x^0\|.
		\end{eqnarray*}
		Последнее выражение при $\|C\| < 1$ стремится к нулю при $k \rightarrow \infty$. Следовательно, $\|x - x^{k+1}\| \rightarrow 0$, т.е. итерационный метод сходится.
		\smallskip
		
		\item \textit{Каким следует выбирать итерационный параметр $\tau$ в методе простой итерации для увеличения скорости сходимости? Как выбрать начальное приближение $x^0$?}
		\smallskip
		
		Каноническая форма одношагового итерационного процесса
		\begin{equation}
	B_{k+1} \sfrac{x^{k+1}-x^k}{\tau_{k+1}}+Ax^k=b,\quad k=0,1,2,\dots.
	\label{iterproc}
		\end{equation}
	
		\textbf{Теорема Самарского.} Пусть $A$ --- самосопряженная положительно опредделенная матрица, $B-\sfrac{\tau}{2}A$ --- положительно определнная матрица, $\tau$ --- положительное число. Тогда при любом выборе нулевого приближения $x^0$ итерационный процесс $\ref{iterproc}$ сходится к решению системы $Ax=b$. 
		\[
		B-\sfrac{\tau}{2}A>0 \Leftrightarrow
		(Bx,x)>\sfrac{\tau}{2}(Ax,x),\quad x \ne 0.
		\]
		Отсюда следует
		\[
		0<\tau<\inf_{x \ne 0} \sfrac{2(Bx,x)}{(Ax,x)}.
		\]
		Тогда для метода простых итераций ($B_{k+1}=E,\, \tau_{k+1} = \tau$)
		\begin{equation}
	0<\tau<\inf_{x \ne 0} 			\sfrac{2(x,x)}{(Ax,x)}=\sfrac{2}{\sup_{x \ne 0}\sfrac{(Ax,x)}{(x,x)}}.
	\label{neravenstvo}
		\end{equation}
		\[
		\sup_{x \ne 0}\sfrac{(Ax,x)}{(x,x)} = \sup_{x \ne 0} \frac{\lambda_1\xi_1^2+\lambda_2\xi_2^2+\dots +\lambda_n\xi_n^2}{\xi_1^2+\xi_2^2+\dots+\xi_n^2} =  \lambda_{1} = \lambda_{max},\quad \lambda_{1} \ge \lambda_{2} \ge \dots \ge \lambda_{n}.
		\]
		В результате из $\ref{neravenstvo}$ следует, что метод простых итераций сходится при $\tau$, принадлежащем интервалу
		\[
		0<\tau<\frac{2}{\lambda_{max}}.
		\]
		
		 Для улучшения скорости сходимости выбирают итерационный параметр $\tau$ так, чтобы выполнялась оценка $\|C\| < 1$ и норма матрицы $C$ была как можно меньше.
		
		Начальное приближение $x^0$ стоит выбирать близким к точному решению. Так как точное решение неизвестно, то общего алгоритма подбора не существует. Только экспериментально можно  повысить сходимость метода, перебирая различные начальные приближения $x^0$.
		
		\item \textit{На примере системы из двух уравнений с двумя неизвестными дайте геометрическую интерпретацию метода Якоби, метода Зейделя, метода релаксации.}
		\smallskip
		% TODO
		
		\item \textit{При каких условиях сходятся метод простой итерации,
			метод Якоби, метод Зейделя и метод релаксации? Какую
			матрицу называют положительно определенной?}
		\smallskip

		\textbf{Теорема.} Пусть $A$ --- симметричная положительно определенная матрица, $\tau>0$ и выполнено неравенство
		\[
		B-0,5 \tau A >0.
		\] 
		Тогда стационарный итерационный метод
		\[
		B \frac{x^{k+1}-x^{k}}{\tau}+Ax^{k}=f
		\]
		сходится.
		
		\textbf{Следствие 1.} Пусть $A$ --- симметричная положительно определенная матрица с диагональным преобладанием, т.е.
		\[
		a_{ii}> \sum_{j \ne i} |a_{ij}|,\quad i =1,2,\dots,n.
		\]
		Тогда метод Якоби сходится.
		
			\textbf{Следствие 2.} Пусть $A$ --- симметричная положительно определенная матрица. Тогда метод верхней релаксации сходится при $0<\omega<2$. В частности, метод Зейделя ($\omega=1$) сходится. Рассматривается действительный случай.
			
			\textbf{Следствие 3.} Метод простой итерации сходится при $\tau<\sfrac{2}{\lambda_{max}},$ где $\lambda_{max}$ --- максимальное собственное значение симметричной положительно определенной матрицы $A$.
			
			Матрица $A$ является положительно определенной, если она удовлетворяет любому из следующих равнозначных критериев:
			\begin{enumerate}
				\item Все собственные значения матрицы $A$ положительны;
				\item \mbox{Определители всех угловых миноров положительны (Критерий Сильвестра)};
				\item $(Ax,x)>0,\quad \forall x \ne 0$.
			\end{enumerate}
		\smallskip
		
		\item \textit{Выпишите матрицу $C$ для методов Зейделя и релаксации.}
		\begin{eqnarray*}
			& (D + \omega L)\frac{x^{k+1} - x^k}\omega + A x^k = b; \\
			& (D + \omega L) x^{k+1} - (D + \omega L) x^k + \omega A x^k = \omega b; \\
			& (D + \omega L) x^{k+1} - (D + \omega L - \omega A) x^k = \omega b;\\
			& (D + \omega L) x^{k+1} - (D - \omega (A - L)) x^k = \omega b.
		\end{eqnarray*}
		Т.к. $A = L + D + U$, то $A - L = D + U$.
		\begin{eqnarray*}
			& (D + \omega L) x^{k+1} - (D - \omega D - \omega U)) x^k = \omega b; \\
			& (D + \omega L) x^{k+1} = ((1-\omega) D - \omega U)) x^k + \omega b; \\
			& x^{k+1} = (D + \omega L)^{-1}  ((1-\omega) D - \omega U)) x^k + \omega (D + \omega L)^{-1} b.
		\end{eqnarray*}
		Итого для метода релаксации $C = (D + \omega L)^{-1}  ((1-\omega) D - \omega U))$. Для метода Зейделя ($\omega = 1$) $C = -(D + L)^{-1} U$.
		\smallskip
		
		\item \textit{Почему в общем случае для остановки итерационного
			процесса нельзя использовать критерий $\|x^k-x^{k-1}\|<\varepsilon?$}
		\smallskip
		
		В общем случае данный критерий останова неприемлем. Метод может медленно сходится и, достигнув заданной точности, найденное приближенное решение будет находится далеко от точного.
		
		\item \textit{Какие еще критерии окончания итерационного процесса
			Вы можете предложить?}
		\smallskip
		
		Ниже приведены следующие критерии останова
		\[
		\|x^{k+1}-x^{k}\| \le \varepsilon,
		\quad 
		\|x^{k+1} - x^k\| \le \varepsilon \|x^k\| + \varepsilon_0,
		\quad
		\left\|\frac{x^{k+1} - x^k}{|x^k| + \varepsilon_0} \right\| \le \varepsilon.
		\]
		Указанные условия прерывания итерационного процесса оперируют нормой изменения численного решения за одну итерацию. Иногда это приводит к неверному заключению о сходимости метода, если, например, метод очень медленно сходится.
		
		В этом случае может оказаться успешным применение другого критерия останова, связанного с нормой невязки (критерий по невязке)
		\[
		\|A x^{k+1} - f\| \le \varepsilon.
		\]
		В случае малости нормы оператора $A$ данный критерий также может оказаться неприемлемым.
		
		Также существуют следующие критерии останова для итерационных процессов
		\[
		\|x^{k+1} - x^k\| \le \dfrac{1-\|C\|}{\|C\|} \varepsilon,
		\qquad
		\|x^{k+1} - x^k\| \le \dfrac{1-\|C\|}{\|C_U\|} \varepsilon
		\]
		

		
		
	\end{enumerate}
	\newpage
	\section{Тест 1}
	Начальная точка --- $(5.0, 5.0, 5.0, 5.0)$.
	
	Критерии останова:
	\begin{enumerate}
			\item $\|x^{k+1} - x^k\| \le \varepsilon$;
		\medskip
		\item $\|x^{k+1} - x^k\| \le \dfrac{1-\|C\|}{\|C\|} \varepsilon$;
		\medskip
			\item $\|A x^{k+1} - f\| \le \varepsilon.$
	\end{enumerate}
%TODO 
%Одной табличкой не получится
%Не знаю как выравнять "Метод по центру"
	\begin{table}[h]
	\caption{Результаты исследования итерационных методов при $\varepsilon=10^{-4}$}
	\footnotesize
	\begin{tabular}{|p{4.3cm}|p{2.4cm}|p{1.9cm}|p{3cm}|p{3.15cm}|}
		\hline
		Метод & $\|C\|$& Оценка для  числа итераций $k_{est}$& Норма ошибки после $k_{est}$ итераций & Число итераций, необходимых для получения решения с точностью $\varepsilon$  \\
		\hline
		Простой итерации $\tau=0.05$ &0.95000000&273&0.00000000&23\\
		\hline
		Простой итерации $\tau= 0.02$ & 0.98000000&691&0.00000000&67\\
		\hline
		Якоби &0.90909091&147&0.00000000&17\\
		\hline
		Зейделя &0.80000000&60&0.00000000&11\\
		\hline
		Релаксации $\omega=1.1$ &0.98000000&779&0.00000000&12\\
		\hline
		Релаксации $\omega=0.5$ &0.90000000&126&0.00000000&32\\
		\hline
	\end{tabular}
\end{table}

	\begin{table}[h]
	\caption{Критерии останова при $\varepsilon=10^{-4}$}
	\footnotesize
	\begin{tabular}{|p{4.3 cm}|p{1.49cm}|p{1.7cm}|p{1.49cm}|p{1.7cm}|p{1.49cm}|p{1.7cm}|}
		\hline
		 	\multirow{2}{4em}{Метод} &\multicolumn{2}{c|}{Критерий останова 1} & \multicolumn{2}{c|}{Критерий останова 2} & \multicolumn{2}{c|}{Критерий останова 3} \\
\cline{2-7}
	&Число итераций&Норма ошибки&Число итераций&Норма ошибки&Число итераций&Норма ошибки\\
		\hline
		Простой итерации $\tau=0.05$ &23&0.00007840&28&0.00000331&27&0.00000650\\
		\hline
		Простой итерации $\tau= 0.02$ &58&0.00050880&80& 0.00000840&79&0.00001020\\
		\hline
		Якоби &18&0.00002823&21&0.00000314&21&0.00000314\\
		\hline
		Зейделя &12&0.00002569&13& 0.00000888&13&0.00000888\\
		\hline
		Релаксации $\omega=1.1$ &13&0.00001613&17&0.00000024&14& 0.00000566\\
		\hline
		Релаксации $\omega=0.5$ &29&0.00024752&36&0.00002396&39&0.00000881\\
		\hline
	\end{tabular}
\end{table}
\newpage
	
		\section{Тест 2}
			Начальная точка --- $(5.0, 5.0, 5.0, 5.0)$.
		
	Критерии останова:
	\begin{enumerate}
		\item $\|x^{k+1} - x^k\| \le \varepsilon$;
		\medskip
		\item $\|x^{k+1} - x^k\| \le \dfrac{1-\|C\|}{\|C\|} \varepsilon$;
		\medskip
		\item $\|A x^{k+1} - f\| \le \varepsilon.$
	\end{enumerate}
	%TODO 
	%Одной табличкой не получится
	%Не знаю как выравнять "Метод по центру"
	\begin{table}[h]
		\caption{Результаты исследования итерационных методов при $\varepsilon=10^{-4}$}
		\footnotesize
		\begin{tabular}{|p{4.3cm}|p{2.4cm}|p{1.9cm}|p{3cm}|p{3.15cm}|}
			\hline
			Метод & $\|C\|$& Оценка для  числа итераций $k_{est}$& Норма ошибки после $k_{est}$ итераций & Число итераций, необходимых для получения решения с точностью $\varepsilon$  \\
			\hline
			Простой итерации $\tau=0.0071518$ &0.5453609&21&0.00000096&15\\
			\hline
			Простой итерации $\tau= 0.01$ &0.93643000 &188&0.00000333&144\\
			\hline
			Якоби &0.26081395&10&0.00000000&5\\
			\hline
			Зейделя &0.26081395&10&0.00000000&4\\
			\hline
			Релаксации $\omega=1.1$ & 0.38689535&14&0.00000000&6\\
			\hline
			Релаксации $\omega=0.5$ &0.63040698&27&0.00000000&18\\
			\hline
		\end{tabular}
	\end{table}
	
	\begin{table}[h]
		\caption{Критерии останова при $\varepsilon=10^{-4}$}
		\footnotesize
		\begin{tabular}{|p{4.3 cm}|p{1.49cm}|p{1.7cm}|p{1.49cm}|p{1.7cm}|p{1.49cm}|p{1.7cm}|}
			\hline
			\multirow{2}{4em}{Метод} &\multicolumn{2}{c|}{Критерий останова 1} & \multicolumn{2}{c|}{Критерий останова 2} & \multicolumn{2}{c|}{Критерий останова 3} \\
			\cline{2-7}
			&Число итераций&Норма ошибки&Число итераций&Норма ошибки&Число итераций&Норма ошибки\\
			\hline
			Простой итерации $\tau=0.0071518$ &15&0.0000797&16&0.0000380&21&0.00000096\\
			\hline
			Простой итерации $\tau= 0.01$ &110& 0.00124755&146&0.00008051&\large{200}&0.00000134\\
			\hline
			Якоби &6& 0.00000209&5&0.00001976&7&0.00000012\\
			\hline
			Зейделя &5&0.00000053&5& 0.00000053&5&0.00000053\\
			\hline
			Релаксации $\omega=1.1$ &7&0.00000443&6& 0.00002565&8& 0.00000065\\
			\hline
			Релаксации $\omega=0.5$ &18&0.00009734&19&0.00005252&26&0.00000079\\
			\hline
		\end{tabular}
	\end{table}
	
	
	
	
	\newpage
	\begin{thebibliography}{1}
		\bibitem{galanin} \textit{Галанин М.П., Савенков Е.Б.} Методы численного анализа математических\\ моделей. М.: Изд-во МГТУ им. Н.Э. Баумана,	2010. 592 с.
		
		
	\end{thebibliography}
	
	
\end{document}
